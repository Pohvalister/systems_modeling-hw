\documentclass[12pt]{article}

\usepackage[utf8]{inputenc}
\usepackage[russian]{babel}
\usepackage{graphicx}
\usepackage{indentfirst}
\usepackage{booktabs}
\usepackage{amsmath}
\usepackage{tabto}


\graphicspath{{pic/}}

\begin{document}

%\begin{center}
%	\textbf{ФЕДЕРАЛЬНОЕ ГОСУДАРСТВЕННОЕ АВТОНОМНОЕ ОБРАЗОВАТЕЛЬНОЕ УЧРЕЖДЕНИЕ ВЫСШЕГО ОБРАЗОВАНИЯ}\\
%	 \\

%	\textbf{САНКТ-ПЕТЕРБУРГСКИЙ НАЦИОНАЛЬНЫЙ ИССЛЕДОВАТЕЛЬСКИЙ УНИВЕРСИТЕТ ИНФОРМАЦИОННЫХ ТЕХНОЛОГИЙ, МЕХАНИКИ И ОПТИКИ}\\

%\end{center}

\begin{center}
	\LARGE 
	\textbf{Лабораторная работа 3}\\
	Моделирование аналоговых сигналов
	\\[3\baselineskip]
\end{center}

\begin{flushright}
	\large
	Выполнил:\\
	студент гр. P4106\\
	Игнашов Иван Максимович\\
	Вариант 8\\
\end{flushright}

\newpage

 \section*{1. Цель работы}
Изучение вариантов вычисления операций над сигналом.

\subsection*{Порядок работы:}
\begin{enumerate}
	\item Сформировать в ЭВМ анализируемый сигнал 
			\begin{equation}
			s(t) = \begin{cases}
						Atcos(2\pi f_0t + \phi_0) & t \in T[t_1, t_2]\\
						0 & иначе
					  \end{cases}
		\end{equation}
 
	\item Для сигнала вычислить аналитически: 
		\begin{itemize}
			\item производную от сигнала - $g(t) = \frac{ds(t)}{dt}$
			\item определённый интеграл от сигнала на интервале $[t_1, t_2]$ - $I = \int_{t_1}^{t_2} s(t)dt$
			\item энергию сигнала - $ E = \int_{t_1}^{t_2} s^2(t)dt$
			\item среднюю мощность (за длительность сигнала) $P = \frac{1}{t_2-t_1}\int_{t_1}^{t_2} s^2(t)dt$
		\end{itemize}
	\item Вычислить те же характеристики сигнала при помощи приближенных методов (численный метод):
		\begin{itemize}
			\item производную от сигнала - $g_i = \frac{s_{i+1} - s_i}{\delta t}$
			\item определённый интеграл от сигнала на интервале $[t_1, t_2]$ - $I = \delta t \sum_{i=0}^{N-1}\frac{s_i + s_{i+1}}{2}$
			\item энергию сигнала - $ E = \delta t \sum_{i=0}^{N-1} s_i^2$
			\item среднюю мощность (за длительность сигнала) $P = \frac{1}{N}\sum_{i=0}^{N-1} s_i^2$
		\end{itemize}
		
	\item Экспериментально подобрать такой временной шаг, при котором ошибка расчёта энергии сигнала не превышает 1\%

\end{enumerate}

\newpage
 \section*{2. Аналитический расчёт искомых величин}%
 	
 
  \subsection*{Определённый интеграл}
 $I = \int_{t_1}^{t_2} s(t)dt = \int_{t_1}^{t_2} A*t*cos(2\pi*f_0*t + \phi_0)dt = \int_{10^{-5}}^{2*10^{-5}} 10^5 * t * cos(2 * \pi * 10^5 * t + \frac{\pi}{2}) dt$\\
 \\
 $\int 10^5 * t * cos(2 * \pi * 10^5 * t + \frac{\pi}{2}) dt = 10^5 * \int -t*sin(2*10^5\pi t)dt = $ \\
 $ = \frac{tcos(2*10^5\pi t)}{2\pi} - \frac{1}{2\pi}\int cost(2*10^5\pi t)dt = \frac{tcos(2*10^5\pi t)}{2\pi} - \frac{sin(2*10^5\pi t)}{4*10^5\pi^2} + const$\\
 \\
 $I = (\frac{tcos(2*10^5\pi t)}{2\pi} - \frac{sin(2*10^5\pi t)}{4*10^5\pi^2} )|_{10^{-5}}^{2*10^{-5}} = \frac{1}{2*10^5\pi} + (-\frac{sin(u)}{4*10^5\pi^2})|_{2\pi}^{4\pi} = \frac{1}{2*10^5\pi}$\\
 \\
 $I \approx 1.5915 * 10^{-6}$ 
 
  \subsection*{Производная функции}
 $s'(t) = \frac{d}{dt}(10^5tcos(2*10^5\pi t + \frac{\pi}{2})) = -10^5(sin(2*10^5\pi t) + 2*10^5\pi t cos(2*10^5\pi t))$
 
  \subsection*{Энергия сигнала}
 $E = \int_{t_1}^{t_2} s^2(t)dt = \int_{t_1}^{t_2} (A*t*cos(2\pi*f_0*t + \phi_0))^2dt = \int_{10^{-5}}^{2*10^{-5}} 10^{10} * t^2 * cos^2(2 * \pi * 10^5 * t + \frac{\pi}{2}) dt$\\
 \\
 $\int 10^{10} * t^2 * cos^2(2 * \pi * 10^5 * t + \frac{\pi}{2}) dt = 10^{10} \int t^2 sin^2(2*10^5\pi t)dt = 10^{10} \int t^2(\frac{1 - cos(4*10^5\pi t)}{2})dt = 5*10^9 \int (t^2 - t^2cos(4*10^5\pi t))dt = 5*10^9 * (\int t^2dt - \int t^2*cos(4*10^5\pi t)dt) =\dots$\\
 \\
 для $t^2 cos(4*10^5\pi t)$:\\
 $\int fdg = fg - \int gdf$ , где $ f = t^2 (df = 2tdt); dg = \frac{sin(4*10^5\pi t)}{4*10^\pi} (g = \frac{sin(4*10^5\pi t)}{4*10^5\pi}$\\
 \\
 $\dots= (-\frac{12.5*10^3 t^2 sin(4*10^5\pi t)}{\pi} +  25*10^3 \int tsin(4*10^5\pi t)dt + 5*10^9 \int t^2dt =\dots$\\
 \\
  для $t sin(4*10^5\pi t)$:\\
 $\int fdg = fg - \int gdf$ , где $ f = t (df = dt); dg = sin(4*10^5\pi t) (g = -\frac{cos(4*10^5\pi t)}{4*10^5\pi}$\\
 \\
 $\dots= -\frac{12.5*10^3 t^2 sin(4*10^5\pi t)}{\pi} - \frac{tcos(4*10^5\pi t)}{16\pi^2} + \frac{1}{16\pi^2} \int cos(4*10^5\pi t)dt + 5*10^9 \int t^2dt$\\
 \\Получается:
 \\
 $E = -\frac{1}{16*10^5\pi^2} + \frac{1}{16\pi^2}\int_{10^{-5}}^{2*10^{-5}} cos(4*10^5\pi t)dt + 5*10^9 \int_{10^{-5}}^{2*10^{-5}} t^2dt = -\frac{1}{16*10^5\pi^2} + \frac{1}{64*10^5\pi^3} \int_{4\pi}^{8\pi} cos(u)du + 5*10^9 \int_{10^{-5}}^{2*10^{-5}} t^2dt$ \\
 \\
 $E = -\frac{1}{16*10^5\pi^2} + \frac{5*10^9t^3}{3} |_{10^{-5}}^{2*10^{-5}} = \frac{7}{6*10^5} - -\frac{1}{16*10^5\pi^2} = \frac{56\pi^2 - 3}{48*10^5\pi^2} \approx 0.000011603$
 
 \subsection*{Средняя мощность}
 
 $P = \frac{1}{t_2 - t_1}E$
 $P = 10^5 * \frac{56\pi^2 - 3}{48*10^5\pi^2} \approx 1.1603$
 
\newpage
 \section*{3. Символьное вычисление характеристик сигнала в MatLab}

 

\newpage
 \section*{4. Пример расчёта через дискретный эквиваленит}
 
\newpage
 \section*{5. Выводы}
 Таблица сравнения методов вычисления:
 \begin{center}
 	\begin{tabular}[c]{l|ccc}
 		Операция & Аналитический & Автоматизированный аналитический & Численный \\\hline
 		&&&\\
 		Интеграл & & & \\
 		Энергия & & & \\
 		Мощность & & & \\
 		
 	\end{tabular}
 \end{center}
 
 
Целью данной лабораторной работы было изучение алгоритмов получения на ЭВМ чисел с заданным законом распределения и построения гистограмм.\\

В процессе выполнения был реализован способ генерации случайных чисел с эрланговским законом распределения.\\

Для выборок, полученных данным генератором были построены гистограмма и графики зависимости значений мат. ожидания и дисперсии от объема выборки.

\end{document}